Šiuo darbu siekama išsiaiškinti kintančios akcijų rinkos pateikiamus duomenis, juos išanalizuoti ir pateikti varotojui patraukliu būdu. Sukurta programėlė bus prieinama vartotojams turintiems įrenginius naudojančius „Android“ operacinę sistemą, kadangi didžioji dalis mobiliųjų įrenginių vartotojų naudoja būtent šią operacinę sistemą. Šio kursinio darbo metu iškeltas uždavinys buvo sukurti sistemą, kuri leistų stebėti realiu laiku kintančias akcijas. Tam, kad galėčiau įgyvendinti tokį uždavinį norėjau išanalizuoti kuo daugiau skirtingų programėlių ir suprasti kuo jos skiriasi, bei ko trūksta, kad jos būtų geresnės ir pasinaudojus tokiais atradimais sukurti sistemą, kuri būtų paprasta naudoti, bet joje netrūktu informatyvaus turinio ir funkcionalumo. Buvo nuspręsta šią programėlę įgyvendinti sukuriant du akcijų atvaizdavimo būdus: programėlėje įkoduotų akcijų trumpinių atnaujinimą iš išorinių šaltinių ir vartotojų asmeniniuose portfeliuose pridėtų akcijų atnaujinimą ir gavimą. Du skirtingi duomenų atvaizdavimo tipai yra išskirti todėl, kad ne visi vartotojai nori matyti iš anksto nustatytų akcijų gavimą realiu laikų, o nori pridėti savo individualias akcijas ir matyti jų pasikeitimus asmeniniuose portfeliuose. Pagrindinė problema, su kuria susidūriau kurdamas sistemą - tai duomenų atvaizdavimas, kuris būtų intuityvus vartotojui, kuris nieko nežino apie akcijų rinką.

Darbo tikslas - sukurti automatizuotą akcijų stebėjimo sistemą, kuri gautų akcijas iš viešai platinamų tarnybų ir jas tiek įkoduotų akcijų trumpinių sąraše, tiek vartotojų asmeniniuose portfeliuose. Darbo uždaviniai - išanalizuoti panašias programėles, suprasti jų turinį, dizaino sprendimus, teikiamą naudą ir pagrindines sistemų savybes. Sudaryti teorinį sistemos modelį aprašant pradines sistemos dizaino ir funkcionalumo dalis. Aprašyti rastas tinkamiausias technologijas programėlės kūrimui, bei jas panaudoti sistemos įgyvendinimo procese.

Pasiekti rezultatai galutininėje programėlėje: sukurta funkcionuojanti akcijų stebėjimo sistema ir vartotojų akcijų portfeliai. Šiuose sąrašuose esančias akcijas galima praskleisti ir pažiūrėti akcijų istorinę kainų kitimo raidą grafiniu formatu, bei planetsnius akcijos duomenis. Taip pat sukurtas valiutos keitimo funkcionalumas, leidžiantis vartotojams iš skirtingų valstybių išsiversti valiutų kainas į sau suprantamą valiutą.