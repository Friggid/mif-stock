Šiuo tyrimu siekama išsiaiškinti kintančios akcijų rinkos pateikiamus duomenis, juos išanalizuoti ir pateikti varotojui patraukliu būdu. Informacija yra prieinama tik vartotojams turintiems įrenginius naudojančius Android operacinę sistemą, kadangi didžioji dalis mobiliųjų įrenginių vartotojų naudoja būtent šią operacinę sistemą. Daugeliui žmonių akcijų rinka atrodo, kaip sudėtingas procesas, reikalaujantis daug pastangų norint išsiaiškinti veikimo ir pasisekimo principus. 

Ši sistema yra pagrįsta informacijos paprastumu žmogui, kuris nėra susipažinęs su akcijų rinka. Sistemos vartotojai gali susikurti savo akcijų portfelį, kuriame gali įdėti pasirinktų įmonių akcijas ir stebėti jų pakitimus per pasirinktą laikotarpį. Taip pat sistema pateikia galimybę plačiau susipažinti su akcijų rinka žaidimo būdu. 

Vartotojas gali investuoti į virtualią akcijų rinką naudodamasis virtualia programėlės valiuta ir stebėti akcijos rinkų pokyčius. Tokiu būdu vartotojas mato ar uždirba arba praranda savo investuotus pinigus ir gali akcijas toliau laikyti arba parduoti. Ši simuliacija leidžia varototojui susipažinti su rinkos veikimu nenaudojant tikrų pinigų. Taip programėlė suteikia edukacinio pobūdžio medžiagą.