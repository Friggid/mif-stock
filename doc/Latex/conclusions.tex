Šiame darbe buvo siekiama sukurti programėlę, kuri vartotojams suteiktų informaciją apie naujausius akcijos rinkų kainų pokyčius. Pagrindinis programėlės tikslas buvo suprogramuoti akcijų kainų stebėjimo sistemą, kur vartotojai patys galėtų pasirinkti norimas akcijas pateikę akcijos simbolį. Taip pat buvo siekiama, kad vartotojai turėtų papildomą valiutų keitimo funkcionalumą, bei galėtų detaliau apžiūrėti kiekvienos akcijos duomenis ir grafinius akcijos kainų pakitimus. Sukurtos programėlės funkcionalumai yra:
\begin{itemize}
	\item Vartotojo akcijų portfelio kūrimas pasirenkant norimą akcijos simbolį.
	\item Programėlėje įkoduotų akcijų simbolių atvaizdavimas.
	\item Akcijų sąrašų atnaujinimas ir išsaugojimas į mobilaus įrenginio atmintį.
	\item Akcijos kainos pasikeitimų per nustatytą laiką grafinis atvaizdavimas ir detalus akcijos duomenų atvaizdavimas.
	\item Valiutų keitimas.
\end{itemize}
Rekomendacijos tolesniam darbo vystymui:
\begin{itemize}
	\item Apjungti išorinius duomenų tiekėjus į vieną, tam, kad nebūtų duomenų paklaidų ir būtų optimizuotas paieškos, bei atvaizdavimo greitis.
	\item Sukurti interaktyvų akcijų pirkimo ir pardavimo žaidimą.
\end{itemize}
